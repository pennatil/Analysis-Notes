\chapter{Differentialrechnung in $\mathbb{R}^n$}
\section{Partielle Ableitungen und Differential}
Wie kann man die Begriffe der \todo{Missing content?? page 113 top} Differentialrechnung auf Funktionen $f:\Omega \subset \mathbb{R}^n\rightarrow\mathbb{R}$ erweitern?\\

Funktion in mehreren variablen sind ein bisschen komplizierter als Funktionen in einer variable.
\subsubsection*{Beispiel}
\begin{enumerate}
\item $f(x)=x^2+5$ ist in ursprung stetig da $\lim\limits_{x\rightarrow 0}f(x)=f(0)$. Aber $f:\mathbb{R}^2\rightarrow\mathbb{R}$ \[f(x,y) = \left\{ {\begin{array}{*{20}{c}}
{\frac{{xy}}{{{x^2} + {y^2}}}}&{(x,y)\not  = (0,0)}\\
0&{(x,y) = (0,0)}
\end{array}} \right.\] ist im Ursprung nicht stetig. %page 114 top
\end{enumerate}