\begin{itemize}
\item[3.] \[\frac{2}{\left( 1-x\right)^3}=\sum\limits_{n=2}^\infty n(n-1)\cdot x^{n-2},\hspace{5mm \abs{x}<1}\]
\end{itemize}

\begin{definition}{5.31}
\begin{enumerate}
\item $f:\Omega\to\R$ ist $m-$mal differenzierbar, falls $\underbrace{\left( \left( f'\right)'\dots\right)'}\limits_{m-\text{mal}}$ existiert. Die $m-$te iterierte Ableitung wird mit
\[f^{(m)}=\frac{\partial^m f}{\partial x^m}:\Omega\to\R\]
bezeichnet. Es gilt: $\forall k,l\geq 0$, $f^{\left(k+l\right)} = \left( f^{(k)}\right)^{\left(l\right)}$
\item $f$ ist in $C^m\left( \Omega\right)$, falls $f$ $m-$mal differenzierbar ist, und alle Funktionen $f=f^{(0)},f', f^{(2)}, \dots, f^{(m)}$ sind stetig
\item $f$ ist in $C^{\infty}\left( \Omega\right)$, falls $f\in C^m \left( \Omega\right)$, $\forall m \geq 0$
\end{enumerate}
\end{definition}

\subsubsection*{Korollar 5.32}
Unter den Voraussetzungen von Satz 5.29\todo{Add reference} ist $f(x)=\sum\limits_{n=0}^\infty a_nx^n$ in $C^\infty\left( -\rho, \rho\right)$ und die Ableitungen von $f$ erhält man durch gliedweises differenzieren.
\subsubsection*{Formel}
\begin{align*}
f(x)&=\sum\limits_{n=0}^\infty a_nx^n, x\in\left( -\rho,\rho\right)\\
f'(x)&=\sum\limits_{n=0}^\infty na_nx^n\\
\vdots\\
f^{(k)}(x)&=\sum\limits_{n=0}^\infty a_n(n)\cdot(n-1)\dots(n-k+1)\cdot x^{n-k}\\
\Aboxed{f^{(k)}(x)&=\sum\limits_{n=k}^\infty a_n\frac{n!}{\left( n-k\right)!} x^{n-k}}
\end{align*}

\section{Taylorformel}
Wir beginnen, als Motivation, mit Polynomen. Sei
\[P(x)=a_0+a_1x+a_2x^2+\dots+a_nx^n\]
ein Polynom, grad $P\leq n$. Durch $x=(x-a)+a$ und Umordnen nach Potenzen von $(x-a)$ erhalten wir
\[p(x)=b_0+b_1(x-a)+b_2(x-a)^2+\dots+b_n(x-a)^n\]

\subsubsection*{Beispiel}\todo{Maybe add big brackets around example? page 233 middle}
$p(x)=x^3+x+1$. Sei $a=1$:
\begin{align*}
p(x)&=\left( \left( x-1\right)+1\right)^3 +\left( x-1\right)+1+1\\
&=\left( x-1\right)^3+3\left( x-1\right)^2+\left( x-1\right)+1+\left( x-1\right)+2\\
&=\left( x-1\right)^3+3\left( x-1\right)^2+4\left( x-1\right)+3
\end{align*}

Wir bestimmen jetzt die Koeffizienten $b$:
\begin{align*}
p(a)&=b_0\\
p'(x)&=b_1+2b_2(x-a)+\dots+nb_n(x-a)\Rightarrow p'(a)=b_1\\
p''(x)&=2b_2+6b_3(x-a)+\dots+n(n-1)(x-a)^{n-2}\Rightarrow p''(a)=2b_2\\
\end{align*}
Rekursiv: $p^{(j)}(a)=j!b_j$ mit der Konvention $b_j=0$ für $j\geq n+1$ da $p^{(n+1)}=0$. Es folgt:
\[p(x)=p(a)+p'(a)(x-a)+\frac{p''(a)}{2!}(x-a)^2+\dots+\frac{p^{(n)}(a)}{n!}\left( x-a\right)^n\]

\subsubsection*{Beispiel}
\begin{align*}
p(x)&=x^3+x+1 \hspace{5mm}p(1)=3\\
p'(x)&=3x^2+1\hspace{8.5mm} p'(1)=4\\
p''(x)&=6x \hspace{15.2mm} p''(1)=6, \frac{p''(1)}{2!}=3\\
p'''(x)&=6 \hspace{16mm} p'''(1)=6, \frac{p'''(1)}{3!}=1\\
\end{align*}
\[p(x)=3+4(x-1)+3(x-1)^2+(x-1)^3\]
Folgendes ist dann naheliegend

\subsubsection*{Satz 5.33}
Sei $\lbrack a,b\rbrack\subset \Omega\subset\R$ und $f\in C^{m-1}\left( \Omega\right)$, $m-$mal differenzierbar auf $\Omega$. Dann gibt es $c\in(a,b)$:
\[f(b)=f(a)+f'(a)(b-a)+\dots+f^{m-1}(a)\frac{(b-a)^{m-1}}{(m-1)!}+\frac{f^m(c)}{m!}(b-a)^m\]

\begin{beweis}{}
Wir betrachten die Funktion
\[g(x)=f(x)+f'(x)(b-x)+\dots+\frac{f^{m-1}(x)(b-x)^{m-1}}{(m-1)!}+\frac{K(b-x)^m}{m!}-f(b)\]
die auf $\Omega$ differenzierbar ist. Dann ist $g(b)=0$. Wähle $K$ so dass $g(a)=0$. Dann gibt es $c\in (a,b)$ mit $g'(c)=0$. Wir berechnen jetzt $g'(x)$:
\[g'(x)=f'(x)+\dots \left( \frac{f^{(j)}(x)(b-x)^j}{j!}\right)'+\dots+K\frac{m(b-x)^{m-1}}{m!}\]
Nun ist
\[\left( \frac{f^{(j)}(x)(b-x)^j}{j!}\right)'=\frac{f^{(j)}(b-x)^{j-1}}{(j-1)!}(-1)+\frac{f^{(j+1)}(x)(b-x)^{j}}{j!}\]
Woraus folgt:

\begin{align*}
g'(x)=&f'(x)\\
&+\left\{ -f'(x)+\frac{f^{(2)}{(x)(b-x)}}{1!}\right\}\\
&+\left\{ -\frac{f^{(2)}(x)(b-x)}{1!}+ \frac{f^{(3)}(x)(b-x)^2}{2!} \right\}\\
&\hspace{2mm}\vdots\\
&+\left\{ -\frac{f^{(m-1)}(x)(b-x)^{(m-2)}}{(m-2)!}+ \frac{f^{(m)}(x)(b-x)^{m-1}}{(m-1)!} \right\}\\
&-K\frac{(b-x)^{m-1}}{(m-1)!}
\end{align*}
Also: \[g'(x)=\left[ f^{(m)}(x)-K\right]\frac{(b-x)^{m-1}}{(m-1)!}\]
Aus $g'(c)=0$ folgt $\boxed{f^m(c)=K}$ und $g(a)=0$ nimmt folgende Form an:
\[g(a)=0=f(a)+f'(a)(b-a)+\dots+\frac{f^{m-1}(c)(b-a)^{m-1}}{(m-1)!}+\frac{f^m(c)}{m!}(b-a)^m-f(b)\]
\end{beweis}

\subsubsection*{Korollar 5.34}
Sei $f:(c,d)\to\R$ $m-$mal differenzierbar, seien $x_0,x\in (c,d)$. Dann gibt es $\xi\in (x_0,x)$ mit
\begin{align*}
f(x)=&f(x_0)+f'(x_0)(x-x_0)\\
&\vdots \\
&+f^{m-1}(x_0)\frac{(x-x_0)^{m-1}}{(m-1)!}\\
&+\frac{f^m(\xi)(x-x_0)^m}{m!}
\end{align*}
Wir führen folgende Terminologie ein:
\[T_mf(x;x_0) = f(x_0)+\dots+f^{(m)}(x_0)\frac{(x-x_0)^m}{m!}\]
ist das Taylor Polynom $n-$ter Ordnung. $\left( f\in C^m\right)$ und
\[R_mf(x;x_0):=f(x)-T_mf(x;x_0)\]
ist der Restterm. \\

Falls $f(m+1)-$mal differenzierbar in $\Omega$ ist, so ist
\[R_mf(x;x_0)=f^{(m+1)}(\xi)\frac{\left( x-x_0\right)^{m+1}}{(m+1)!}, \xi\in\left( x_0,x_0\right)\]

\subsubsection*{Bemerkung}
Bei der Differenzierbarkeit haben wir gesehen, dass die lineare Funktion $f'(a)+f'(a)(x-a)$ im folgenden Sinne eine gute Approximation der Funktion $f(x)$ darstellt: Es gilt
\[f(x)=f(c)+f'(a)(x-a)+R_1f(x;a) = T_1f(x;a)+R_1f(x;a)\]
und
\[\lim\limits_{x\to a}\frac{R_1f(x;a)}{x-a} = \lim\limits_{x\to a}\left(\frac{f(x)-f(a)}{x-a}-f'(a)\right)=0\]
Die Taylorformel gibt nun an, wie mann diese Approximation noch verbessern kann.
\[f(x) = \underbrace {{T_m}f(x;a)}_{{\text{Approximation}}} + \underbrace {{R_m}f(x;a)}_{{\text{fehler}}}\]
Für diesen Fehler gilt:
\[\lim\limits_{x\to a}\frac{R_mf(x;a)}{(x-a)^m}=0\tag{\textasteriskcentered}\]
Das bedeutet, dass wenn $x$ nahe bei $a$ ist, $R_mf(x;a)$ klein ist im Vergleich zu der schon sehr kleinen Grösse $(x-a)^m$. Für den Restterm $R_mf(x;a)$, haben wir die Abschätzung \todo{not sure where the additional content goes, page 240 middle}
\[R_mf(x;a) = f(x)-T_mf(x;a) = \left[ f^m(\xi) -f^m(a)\right]\frac{(x-a)^m}{m!}\]

\[\abs{R_mf(x;a)}<\sup\limits_{a<\xi<x}\abs{f^{(m)}(\xi)-f^m(a)}\frac{(x-a)^m}{m!}\]
Falls $f\in C^{m+1}$, können wir denn Mittelwertsatz anwenden. Dann folgt:
\begin{align*}
\abs{R_mf(x;a)}&\leq\left( \sup\limits_{a<\xi<x}\abs{f^{(m+1}(\xi)}\right)\abs{x-a}\frac{(x-a)^m}{m!}\\
&\leq M\frac{(x-a)^{m+1}}{m!}\\
\Rightarrow & \abs{\frac{R_mf(x-a)}{(x-a)^m}}\to 0, x\to a
\end{align*}

\todo[inline]{Begin additional content}
\[f(x) = f(a)+\dots+\frac{f^{m-1}(a)(x-a)^{m-1}}{(m-1)!}+\frac{f^{m}(\xi)(x-a)^{m}}{(m)!}\]
\begin{align*}
f(x) + \frac{f^{m}(a)(x-a)^{m}}{(m)!} &= T_m f(x;a)+\frac{f^{m}(\xi)(x-a)^{m}}{(m)!}\\
\Rightarrow f(x)-T_mf(x;a) &= \frac{f^m (\xi) -f^m(a)}{m!}(x-a)^m
\end{align*}
\todo[inline]{End additional content}

\subsubsection*{Beispiel 5.35}
\begin{center}
\begin{tabular}{r l}
$\begin{aligned}
f(x)&=\sin(x)\\
f'(x)&=\cos(x)\\
f''(x)&=-\sin(x)\\
f'''(x)&=-\cos(x)\\
f^4(x)&=\sin(x)\\
f^5(x)&=\cos(x)\\
\end{aligned}$&
$\begin{aligned}
x_0&=0\\
f'(0)&=1\\
f''(0)&=0\\
f'''(0)&=-1\\
f^4(0)&=0\\
^5(0)&=1\\
\end{aligned}$
\end{tabular}
\end{center}
Also
\begin{align*}
\sin x &= x-\frac{x^3}{3!}+\frac{\cos(\xi)}{5!}x^5\\
T_1(x)&=x=T_2(x)\\
T_3(x)&=x-\frac{x^3}{3}=T_4(x)
\end{align*}
Insbesondere
\[\abs{\sin(x)-x+\frac{x^3}{3}}\leq \frac{x^5}{5!}\]
liefert für ein kleines $\abs{x}$ eine sehr gute Approximation von, zum Beispiel, $x=\frac{1}{10}$
\[\abs{\sin\left( \frac{1}{10}\right)-\frac{1}{10}+\frac{1}{1000}}<\frac{1}{10^5\cdot 5!}\]
\[\sin\left( \frac{1}{10}\right)\approx \frac{1}{10}-\frac{1}{1000}=0.1-0.001=0.099\]

\subsection*{Lokale Extrema}
Wir haben den folgenden Satz schon gesehen:

\subsubsection*{Satz 5.12}
Sei $f:\lbrack a,b\rbrack\to\R$ stetig und auf $(a,b)$ differenzierbar. Sei $z_+\in( a,b)$ mit $f(z_+)=\max\left\{ f(x):x\in\lbrack a,b\rbrack\right\}$. Dann gilt $f'(z_+)=0$\\

\noindent Mittels Taylorformel können wir lokale Extrema (Maxima und Minima) diskutieren. \\

\noindent Sei $\Omega\subseteq\R$, $f:\Omega\to\R, x_0\in\Omega$

\begin{definition}{5.38}
\begin{enumerate}
\item $x_0\in\Omega\subset\R$ heisst lokale Maximalstelle (bzw. Minimalstelle) von $f$, falls es $r>0$ gibt s.d. $\forall x\in B_r(x_0)$, $f(x)\leq f(x_0)$ (resp. $f(x)\geq f(x_0)$, $\forall x\in B_r(x_0)$). Die lokale Minimalstelle (bzw. Maximalstelle) ist strikt, falls $f(x)>f(x_0)$ (bzw. $f(x)<f(x_0)$)
\end{enumerate}

\end{definition}